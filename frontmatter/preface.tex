\chapter*{Preface}

For as long as I can remember, I have dreamed of writing a book where no step is skipped, where every concept finds its place in a logical chain, and where the reader is never left with the feeling that something essential has been hidden between the lines. Too often, in mathematics and in the sciences, explanations assume too much, or leave gaps that only those with prior training can fill. This book is my answer to that problem.

When I began teaching and researching neural networks, I was struck by how fragmented the explanations often were. Some texts went straight to the formulas, as if the reader should already know why those symbols mattered. Others remained on the surface, offering analogies without showing the rigorous backbone of mathematics that sustains the theory. The result is that many students, researchers, or professionals are left either overwhelmed by abstraction or undernourished by oversimplification. My conviction is that knowledge should not be fragmented: it should flow as a continuous path where each stone is visible and firmly placed for the next step.

That is the spirit of this book. Here, I aim to construct a narrative where the reader walks hand in hand with the mathematics, never jumping over a void of reasoning. Concepts are not presented as isolated facts, but as elements of a larger structure, built slowly with logic as mortar. The aim is not just to show \emph{what} neural networks are, but to reveal \emph{why} they are built the way they are, and how mathematics---linear algebra, calculus, probability, optimization---becomes the invisible skeleton that gives them life.

This is a book for a wide audience. It is written for the undergraduate student taking their first steps in artificial intelligence, for the graduate researcher who needs a rigorous reference, and for professionals in the industry who want to understand the deeper principles behind the tools they use. My hope is that this book will serve as both a guide and a reference: a text you can study systematically from beginning to end, but also one you can revisit to clarify details or to build new insights.

Writing this book has also been a personal journey. I wanted to produce something that stands apart from textbooks that either rush through derivations or leave entire arguments as an exercise for the reader. This will not be a book of shortcuts. It will demand effort. There will be formulas, derivations, and proofs. But I promise that every line is there for a reason. Nothing is left to chance, and nothing is presented without context. If you walk with me through these pages, you will not only learn what neural networks are and how they work---you will also discover the rhythm and poetry of mathematics itself, the logic that binds the abstract to the real.

Finally, this book is also a gesture of trust. Trust in the reader's intelligence, in their patience, and in their hunger for depth. I have chosen not to underestimate you, whoever you are, because I believe that clarity is not about removing complexity but about guiding through it.

May these pages serve as a bridge: between mathematics and intuition, between logic and imagination, and between the academic world and real-world applications.
