\chapter{Convolutional Neural Networks (CNNs)}
\section{Mathematical Basis of Convolution}
\subsection{Discrete convolution definition}
\subsection{Cross-correlation vs convolution}
\subsection{Properties of convolution (linearity, shift-invariance)}
\subsection{2D convolution for image data}
\subsection{Backpropagation through convolutional layers}

\section{Feature Maps, Receptive Fields, Pooling}
\subsection{Concept of receptive field}
\subsection{Feature map construction and interpretation}
\subsection{Stride, padding, and dilation}
\subsection{Pooling operations (max, average, global)}
\subsection{Effect on translation invariance and dimensionality reduction}

\section{Modern CNN Architectures (AlexNet, VGG, ResNet)}
\subsection{AlexNet: deep convolution revival}
\subsection{VGG: simplicity and depth}
\subsection{ResNet: residual connections and very deep networks}
\subsection{Other influential models (Inception, DenseNet)}
\subsection{Trends in modern convolutional design}

\section{Applications in Vision, Audio, and Physics}
\subsection{Image classification and object detection}
\subsection{Semantic segmentation and medical imaging}
\subsection{Audio spectrogram analysis}
\subsection{CNNs for time-series and sensor data}
\subsection{Physics and scientific data modeling}

